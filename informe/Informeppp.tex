\documentclass[a4paper]{article}
\usepackage[spanish]{babel}
\usepackage[utf8]{inputenc}
\usepackage{amsmath}
\usepackage{listings}
\usepackage{xcolor}
\usepackage{amsfonts}
\usepackage{amssymb}
\usepackage[pdftex]{hyperref}
\usepackage{todonotes}

\begin{document}
    \title{\textbf{Proyecto \#1 DAA:} Procastinaci\'on ++}
    \author{L\'azaro Daniel Gonz\'alez Mart\'inez y Alejandra Monz\'on Pe\~na}
    \date{}
    \maketitle

    \section*{Desccripci\'on del problema}
    Sean las cadenas \textbf{S}, \textbf{T} y una cadena vac\'ia \textbf{A}, se construyen nuevas cadenas de la siguiente forma: 

    \begin{itemize}
        \item[$\diamond $] Quitar la primera letra de \textbf{S} y ponerla al inicio de la cadena \textbf{A}.
        \item[$\diamond $] Quitar la primera letra de \textbf{S} y ponerla al final de la cadena \textbf{A}.    
    \end{itemize}

    Estas operaciones se pueden realizar hasta que \textbf{S} quede sin caracteres.\\ 

    Se desea contar cuantas cadenas diferentes se pueden crear con estas operaciones tales que \textbf{T} sea prefijo de 
    la nueva cadena.\\ 

    Definimos por comodidad la construcci\'on de cadenas combinando de cualquier forma posible las dos operaciones permitidas como 
    \textit{construcci\'on por aburrimiento}.

    \section*{Soluci\'on Backtrack}
    Una primera soluci\'on al problema planteado se puede obtener haciendo una exploraci\'on por todas las posibles cadenas que se pueden formar 
    y en cada caso comprobar si \textbf{T} es prefijo de dicha cadena. 

    		% Configuración de Listings
	\lstset{keywordstyle=\color{blue}, basicstyle=\small}

	\begin{figure}[htb]
	
    \begin{lstlisting}[language=Python]

        def Procastinacion(S,T):
            if len(T) > len(S):
                return 0
            return Procastinacion2(S,T,'',0)
        
        def Procastinacion2(S,T,A,i):
            m = len(T)
            k = len(A)
            if len(S) <= i:
                return T == A[0:m]
            return Procastinacion2(S,T,f'{A}{S[i]}',i+1) 
                        + Procastinacion2(S,T,f'{S[i]}{A}',i+1) 
                            + ((m <= k) and (T == A[0:m]))
	\end{lstlisting}
	\caption{Código python del backtrack.}
	\end{figure}

    Este algoritmo de soluci\'on, aunque efectivo es demasiado costoso computacionalmente, sean $|S| = n$ y $|T| = m$, se tiene que:

    \begin{equation*}
        T(n) = 2 T(n-1) + m 
    \end{equation*}

    de donde,al utilizar el Teorema Maestro para funciones decrecientes, se obtiene para este problema que 
     $T(n) = \Theta(2^n + m)$.

    \section*{Explotando caracter\'isticas del problema}
    Una primera mejora que se puede hacer, se basa en la idea de que la primera letra de \textbf{S} 
    al poder colocarse tanto al inicio como al final de la cadena vac\'ia \textbf{A}, a partir de esa decisi\'on 
    se generan dos \'arboles de cadenas exactamente iguales(con las mismas cadenas), solo que estas se pueden considerar '' diferentes '' por 
    haber tenido una decisi\'on inicial distinta.\\ 

    Por tanto una mejora inicial, consiste en no duplicar innecesariamente el espacio de b\'usqueda, sino que asumir que en \textbf{A} inicialmente 
    est\'a el primer caracter de \textbf{S} y duplicar el resultado final. Aunque esta idea no mejora la complejidad temporal, reduce a la mitad la cantidad de operaciones a realizar.

    Al analizar la forma en que se manipulan los caracteres de \textbf{S} para originar nuevos vocablos, se 

    \section*{Soluci\'on con Programaci\'on Din\'amica}

    Representando las cadenas \textbf{S} y \textbf{T} por sus caracteres, tenemos:

    \begin{itemize}
        \item[] \textbf{T} = $T_0T_1 ... T_{m-1}$
        \item[] \textbf{S} = $S_0S_1 ... T_{n-1}$ 
    \end{itemize}

    Entonces $T_i$ ($S_i$) denota al i-\'esimo m\'as un cararacter de \textbf{T}(\textbf{S}), de igual modo 
    $T_{i...j}$($S_{i...j}$) denota a la subcadena $T_iT_{i+1}...T_j$ ($S_iS_{i+1}...S_{j}$) de la cadena \textbf{T} (\textbf{S}).\\
    
    Definiendo la funci\'on $f(i,j)$ como la cantidad de cadenas $A_{i...j}$ \textit{construir por aburrimiento} con los primeros $j-i+1$ 
    caracteres de S (es decir con los caracteres de $S_{0...j-i}$) donde 
    donde para $j < m$, $A_{i...j} = T_{i...j}$ y para $j \geq m $, $A_{i...j} = T_{i...m-1}A_{m...j}$.\\ 
    
    Luego $\sum_{j= m-1}^{n-1}f(0,j)$ es la cantidad total de cadenas que se pueden \textit{construir por aburrimiento} que tienen como prefijo a \textbf{T}.\\
    
    Notemos qu\'e, para $i < m$ se cumple que: %TODO:

    \begin{equation*}
        f(i,i) = \left\{ \begin{aligned}
            &2, &T_i = S_0\\
            &0, &eoc
        \end{aligned} \right.
    \end{equation*}

    Esto se debe a que, como queremos formar subcadenas de tama\~no 1 de \textbf{T} con el primer caracter de \textbf{S}, tenemos en los casos que hay coincidencia, dos maneras de colocar el caracter, (por delante y por detr\'as) y en los restantes casos 
    se tiene 0 puesto que no se tiene ninguna subcadena de \textbf{T} de longitud 1 al tomar ese caracter.\\
    
    Adem\'as se cumple, para $i \geq m$, que $f(i,i) = 2$, puesto que hasta ahora tenemos una posible cadena A de tama\~no $i-1$, con prefijo \textbf{T} y cualquiera sea el pr\'oximo caracter 
    $S_i$ se puede colocar al final de \textbf{A}; el valor es 2 puesto que esta cadena \textbf{A} pudo tener como decisi\'on inicial para su primer cract\'er (el primero que se haya colocado, no necasariamente el que quede en la posici\'on 0) 
    cualquiera de las dos variantes (ponerlo por delante, o por atr\'as).\\ 

    A modo general $f(i,j)$ se puede construir recursivamente como: 

    \begin{equation*}
        f(i,j) = f(i+1,j)\mathbb{I}_{ \{T_i = S_{j-i+1} \vee i \geq m \}}  +  f(i,j-1)\mathbb{I}_{ \{T_j = S_{j-i+1} \vee j \geq m \} } 
    \end{equation*}

    Ya que $S_{j-i+1}$ se puede agregar por delante a las cadenas $A_{i+1}...A_j$, o por atr\'as a las cadenas 
    $A_{i}...A_{j-1}$, cuando $S_{j-i+1}$ coincide con $T_i$ o $T_j$ respectivamente, formando as\'i cadenas $A_{i}...A_{j}$.\\

    En los casos que $j \geq m$, a la subcadena $A_{i...j-1}$ se le puede agregar el caracter $S_{j-i+1}$ por atr\'as sin afectar 
    el prefijo, generando la cantidad de cadenas que hab\'ia en $f(i,j-1)$, pero ahora de la forma $A_{i...j}$.\\
    
    Por \'ultimo cuando $i \geq m$ entonces $j \geq m$ y se cumple lo anterior adem\'as de que ahora se puede agregar $S_{j-i+1}$ como prefijo de 
    $A_{i+1...j}$ y por lo tanto tendr\'iamos $f(i+1,j)$ cadenas m\'as de la forma $A_{i...j}$.
    
    \begin{lstlisting}[language=Python]
def solve_dp(S,T):
    n = len(S)
    m = len(T)
        
    if m > n:
        return 0
        
    dp = [[0 for j in range(n)] for i in range(n)]
    
    for i in range(n):
        if i >= m or T[i] == S[0]:
            dp[i][i] = 2
    
    for k in range(1, n):    
        c = S[k]
        
        i = 0        
        for j in range(k, n):
            if i >= m or c == T[i]:
                dp[i][j] += dp[i+1][j]
            if j >= m or c == T[j]:
                dp[i][j] += dp[i][j-1]            
            i += 1            
    
    return sum(dp[0][m-1:])
    \end{lstlisting}
	
	\section*{Correctitud del algoritmo}
	
	Como nos queda claro que $\sum_{j= m-1}^{n-1}f(0,j)$ es la respuesta a nuestro problema, demostrar que $dp[i][j] = f(i,j)$, sería suficiente para probar la correctitud del algoritmo. Por lo tanto demostremos que en el momento que se actualiza el valor de $dp[i][j]$ este coincidirá con $f(i,j)$.
	
	Hagamos inducción en la longitud de la cadena que estamos formando $A_{i...j}$, lo que es equivalente a hacer la inducción sobre el ciclo en que itera $k$.
	
	\begin{itemize}
		\item Caso base: $|A_{i...j}| = 1$
		
		Como por definición, $i \leq j$ entonces $|A_{i...j}| = 1$ si y solo si $i = j$, ya que $|A_{i...j}| = j-i + 1 = 1$.
		Luego todas las cadenas que puedan pertenecer a nuestro caso base están en la diagonal de $dp$, y se inicializan atendiendo a la definición de $f(i, i)$ dada \todo{Citar arriba}, entonces tenemos que:
		$$dp[i][i] = f(i, i)$$
		Note que se inicializan así en la linea \todo{Citar linea} de nuestro código.
		
		\item Caso Hipótesis: Supongamos que para toda cadena $A_{i^\prime...j^\prime}$ de tamaño menor que $p$ se cumple que $dp[i^\prime][j^\prime] = f(i^\prime, j^\prime)$ con $j^\prime - i^\prime + 1 < p$. Esto es equivalente a que hasta la iteración $p-1$ del ciclo que itera sobre $k$, se cumple que lo planteado anteriormente\footnote{Observe que, para que la fórmula tenga sentido, $i^\prime \leq j^\prime$}.
		
		Sea $A_{i...j}$ de tamaño $p$. Podemos asumir $i<j$, porque el caso base ya está tratado a parte. Luego como
		$$f(i,j) = f(i+1,j)\mathbb{I}_{ \{T_i = S_{j-i+1} \vee i \geq m \}}  +  f(i,j-1)\mathbb{I}_{ \{T_j = S_{j-i+1} \vee j \geq m \} } $$
		se aprecia que $f(i,j)$ depende de los valores de $f(i+1,j)$ y $f(i,j-1)$. Luego como $|A_{i+1...j}| = j-(i+1) + 1 = j-i$, y $j-i < p$, entonces por hipótesis $f(i+1,j) = dp[i+1][j]$. De manera similar como $|A_{i...j-1}| = j-1 - i + 1 = j-i < p$, aplicamos la hipótesis pero en este caso $f(i,j-1) = dp[i][j-1]$.
		Luego
		$$f(i,j) = dp[i+1][j]\mathbb{I}_{ \{T_i = S_{j-i+1} \vee i \geq m \}}  +  dp[i][j-1]\mathbb{I}_{ \{T_j = S_{j-i+1} \vee j \geq m \} } $$
		y finalmente $dp[i][j]$ al actualizarse con este cálculo nos quedará:
		$$dp[i][j] = f(i,j)$$				
	\end{itemize}

	Luego por Principio de Inducción Matemática queda demostrado que al concluir el algoritmo queda computado para cada $dp[i][j]$, con $0 \leq i\leq j < n$, la cantidad de cadenas $A_{i...j}$ que se pueden \textit{construir por aburrimiento} con los primeros $|A_{i...j}| = j-i+1$ caracteres de $S$, es decir, queda guardado en $dp[i][j]$ el valor de $f(i,j)$.	
	
	\section*{Complejidad Temporal}
	
	Definir e inicializar la matriz $dp$ se hace en $2n^2$ operaciones ya que esta tiene dimensión $n \times n$.
	
	El ciclo que itera por $k$ ejecuta el ciclo que itera por $j$, unas $n-1$ veces. Y este ciclo se ejecuta $n-k$ veces para la iteración $k$. Es decir, ambos ciclos ejecutan un total de veces igual a:
	$$ \sum_{k=1}^{n-1} (n-k) = (n-1) + (n-2) + ... + 1 = \frac{n(n-1)}{2} $$
	
	Finalmente cuando se calcula la suma de los $dp[0][i]$ para $i \geq m$, se hacen $n-m + 1$ iteraciones.
	
	Por lo tanto al sumar cada uno de estos costos considerando también, el tiempo constante que requieren las operaciones intermedias, nos quedaría algo así:
	
	$$T(n,m) = c_1 + 2n^2 + c_2\frac{n(n-1)}{2} + n-m+1$$
	
	Demostremos que $T(n,m)$ es $O(n^2)$:
	
	Para esto debemos encontrar una constante $C$ tal que a partir de cierto $n$, se cumpla $T(n,m) \leq Cn^2$.
	$$T(n,m) \leq c_1 + 2n^2 + c_2\frac{n(n-1)}{2} + n + 1$$
	
	Calculemos el límite:
	$$ \lim_{n}\left(\frac{c_1 + 2n^2 + c_2\frac{n(n-1)}{2} + n + 1}{n^2}\right)$$
	$$= \lim_{n}\left(\frac{2c_1 + 4n^2 + c_2n(n-1) + 2n + 2}{2n^2}\right)$$
	$$= \frac{4+c_2}{2}$$
	Como este límite es finito y mayor que $0$, existirá una constante $C$ tal que :
	$$ c_1 + 2n^2 + c_2\frac{n(n-1)}{2} + n + 1 \leq Cn^2$$
	y por transitividad, $ T(n,m) \leq Cn^2 $. Luego $T(n,m)$ es $O(n^2)$.
	
	Incluso si queremos ser más exquisitos podemos demostrar que $T(n,m) = \theta(n^2)$. \\
	
	Como:
	$$c_1 + 2n^2 + c_2\frac{n(n-1)}{2} \leq T(n,m)$$
	Calculando el límite siguiente:
	$$ \lim_{n}\left(\frac{c_1 + 2n^2 + c_2\frac{n(n-1)}{2}}{n^2}\right)$$
	$$= \lim_{n}\left(\frac{2c_1 + 4n^2 + c_2n(n-1)}{2n^2}\right)$$
	$$= \frac{4+c_2}{2}$$
	De forma similar, existirá una constante $C$ tal que a partir de cierto $n$, se cumple que 
	$$Cn^2 \leq c_1 + 2n^2 + c_2\frac{n(n-1)}{2}$$
	y por transitividad se cumple que $Cn^2 \leq T(n,m)$. Luego $T(n,m) = \Omega(n^2)$. Y finalmente $T(n,m)$ es $\theta(n^2)$.
	
	\section*{Generador y Probador de casos}
	
	Para comprobar la correctitud y eficiencia de los algoritmos creamos un generador y probador de casos pruebas. Para la generación de casos definimos 3 alfabetos:
	\begin{itemize}
		\item $A_1 = \{\texttt{a},\texttt{b},\texttt{c}\}$
		\item $A_2 = \{\texttt{a},\texttt{b},\texttt{c},\texttt{d}\}$
		\item $A_3 = \{\texttt{a},\texttt{b},\texttt{c},\texttt{d},\texttt{e}\}$
	\end{itemize}

	Generamos $6$ cadenas $T$, con longitud entre $2$ y $19$. Y por cada $T$ se generaron cadenas $S$ de tamaño desde $|T|$ hasta $10|T|-1$, de $3$ tipos, mezclando aleatoriamente las letras de $T$ con las que se pueden añadir según el alfabeto del tipo seleccionado:
	\begin{itemize}
	\item Tipo 1: Añadiendo caracteres del Alfabeto 1
	\item Tipo 2: Añadiendo caracteres del Alfabeto 2
	\item Tipo 3: Añadiendo caracteres del Alfabeto 3
	\end{itemize}
	La selección aleatoria de los caracteres es uniforme, es decir, cada letra tiene la misma probabilidad de ser seleccionada.
	
	De esta forma se generaron $22680$ casos de pruebas de los que $7404$ resultaron con valor de $0$ al ser evaluadas por el algoritmo de programación dinámica, para representar el $32.6455\%$ de los casos totales. Para comparar ambos algoritmos, se utilizaron $630$ casos de pruebas de los que el $92$ de ellos resultaron en $0$, representando el $14.6031\%$ de estos. Además para los $630$ casos seleccionados, se obtuvo el mismo valor al evaluar ambos algoritmos. La diferencia significativa de ambos algoritmos es en la complejidad temporal, donde analíticamente son muy diferentes. Y en la práctica esto se comprobó donde, los $630$ casos corrieron en menos de $1$ minuto con el algoritmo de programación dinámica. Mientras que con el algoritmo de backtrack, se demoró más de $12$ horas.
	
	El generador, el tester, y el comparador se encuentran en los archivos \texttt{generator.py}, \texttt{tester.py} y \texttt{check.py}, respectivamente.


\end{document}